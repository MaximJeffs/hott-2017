\documentclass[a4paper,12pt]{article}
\usepackage{amsmath}
\usepackage{amsfonts}
\usepackage{amsthm}
\usepackage{enumitem}

\newcommand{\sub}{\subseteq}
\newcommand{\NN}{\mathbb{N}}
\newcommand{\ZZ}{\mathbb{Z}}
\newcommand{\CC}{\mathbb{C}}
\newcommand{\RR}{\mathbb{R}}
\newcommand{\of}{\circ}
\newcommand{\del}{\partial}
\theoremstyle{definition}
\newtheorem*{thm}{Theorem}
\newtheorem*{defn}{Definition}
\newtheorem*{exer}{Exercise}
\newtheorem*{example}{Example}

\title{Homotopy Type Theory Notes: Concepts from Topology}
\author{}
\begin{document}
\maketitle

We start with a crash course in some concepts from topology.

\section{The Fundamental Group}

First we set up some definitions.

\begin{defn}
A \emph{topological space} $(X, \Gamma)$ is a non-empty set $X$ and a collection of subsets of $X$, $\Gamma$, such that $\emptyset, X\in \Gamma$ and $\Gamma$ is closed under arbitrary unions and finite intersections.
We say that $\Gamma$ is a \emph{topology} for $X$, and call the members of $\Gamma$ \emph{open sets}.
\end{defn}

\begin{defn}
Given topological spaces $(X, \Gamma)$ and $(Y, \Delta)$, a function $f: X\to Y$ is \emph{continuous at a point} $x\in X$ if for every $U\in\Delta$ with $f(x)\in \Delta$ (that is, for every open set containing $f(x)$) there is some $T\in \Gamma$ such that $x\in T$ and $f(T)\sub U$.

Then we say $f$ is \emph{continuous} if it is continuous at all $x\in X$.
\end{defn}

\begin{defn}
Let $I$ be the unit interval, $[0, 1]$.
A \emph{path} in $X$ (with an associated topology) is a continuous map $f: I\to X$.
\end{defn}

We assume that any given topological space is connected -- that is, that for all $a, b\in X$, there exists a path $f$ such that $f(0) = a$ and $f(1) = b$.
We also assume all maps are continuous unless otherwise noted.

Two paths $f, g$ are \emph{equivalent} if there is a homotopy between them.
That is, if there exists some $H: I\times I\to X$, such that $H(0, s) = f(s)$, $H(1, s) = g(s)$, and $H$ has fixed endpoints, so $H(t, 0) = a$ and $H(t, 1) = b$ for some $a, b\in X$.

\begin{defn}
A \emph{loop} is a path $f: I\to X$ such that $f(0) = f(1)$.
\end{defn}

Note that we can also write a loop as a function from $(S^1, y)$ to $(X, x)$, with $f(y) = x$ fixing the base point.

Given two paths, with one starting at the endpoint of the other, we can compose them.

\begin{defn}
Given paths $f, g$ with $f(1) = g(0)$, we define $fg: I\to X$ by
$$fg(t) = \begin{cases}f(2t) & t\leq 0.5\\ g(2t-1) & t > 0.5\end{cases}$$
\end{defn}

We can then compose all loops with some fixed base point $x$.
This gives a group structure on the loops with base point $x$, with composition as defined, the inverse of $f$ given by $f^{-1}: t\mapsto f(1-t)$ and identity $e: t\mapsto x$.

\begin{exer}
Show that this is a group: that the identity and inverse are as described, and that composition is associative.
\end{exer}

\begin{defn}
The \emph{fundamental group} or \emph{first homotopy group} of a topological space $X$ and point $x\in X$, $\pi_1(X, x)$, is the group of loops with base point $x$.
\end{defn}

We give two examples of the fundamental group.

\begin{example}
Consider $\pi_1(\RR^2, 0)$.
This topological space has trivial fundamental group, as for any loop $f$ with base point 0, $H(s, t) = sf(t)$ is a homotopy from the trivial loop to $f$.
\end{example}

\begin{example}
The fundamental group of $(S^1, \Gamma)$, the circle, is more interesting.
It is given by the free group on one generator, which is the loop around the circle, so is isomorphic to $\ZZ$.
\end{example}

\section{The Second Homotopy Group}

We also (somewhat more informally) can define the second homotopy group.
Consider the embeddings of $(S^2, y)$ in a topological space with base point $(X, x)$.

Note that we can also write these maps as having domain $I\times I$ with the restriction that the boundary of $I\times I$ is sent to $x$.
Then we can compose two of these maps as follows:

Given two maps $f, g: I\times I\to X$, each sending the boundary of the unit square to $x$, we have
$$fg(x, y) = \begin{cases}
		f(x, 2y) & y\leq 0.5\\
		g(x, 2y-1) & y > 0.5
	\end{cases}$$
The inverse of a map $f: I\times I\to X$ is given by $f^{-1}: (x, y)\mapsto f(1-x, y)$.

Now, unlike $\pi_1$, $\pi_2$ is abelian!
(The Eckmann-Hilton argument shown was heavily picture based so I haven't typed it up well; this is very sketchy and elaboration would be awesome!)
The proof given could in fact be used to construct $\ZZ$ distinct proofs, as it relied on rotating the domain of $f$ around the domain of $g$ within a square, which can be done multiple times.
So in fact, the homotopy type of the number of proofs that $\pi_2$ is abelian is $\ZZ$.

\begin{exer}
Formalise this definition of $\pi_2$, showing the inverse is as claimed, and show it is associative and abelian.
\end{exer}

\begin{example}
As with $\pi_1(S^1)$, $\pi_2(S^2)$ is the free group on one generator (which is the identity map on $S^2$) so is isomorphic to $\ZZ$.
In fact, in general $\pi_n(S^n)$ is the free group with one generator: the identity map on $S^n$.
\end{example}

\begin{example}
Also, $\pi_3(S^2)$ is isomorphic to $\ZZ$.
(Teaser for some future week: this is equivalent to the homotopy type of the number of proofs that $\pi_2$ is abelian being $\ZZ$! By this logic, as $\pi_4(S^3)$ is isomorphic to $\ZZ_2$, there should be two distinct proofs that $\pi_3$ is abelian.)
\end{example}

Another way of conceptualising $\pi_2$ (and an approach that is much easier to deal with algorithmically) is to consider \emph{groupoids}.

\begin{defn}
A \emph{groupoid} is a category where all the morphisms are invertible.

Equivalently, a groupoid is a set with a binary operation where all the elements have right and left labels, and $fg$ is defined only if the right label of $f$ is the same as the left label of $g$.
\end{defn}

(The image I had in my head of how a groupoid works was of paths in a graph -- which I think is basically the same as the category definition -- where you're interested in the structure created by these paths, with operation joining two paths.)

\begin{defn}
The \emph{fundamental 2-groupoid} in a topological space $X$ is a 3-layer object, given as follows:
\begin{enumerate}[start=0]
\item The 0-layer is all points in $X$,
\item The 1-layer is all paths between points (\underline{not} quotiented out by homotopy), and
\item The 2-layer is all paths between the paths in the 1-layer, up to homotopy.
\end{enumerate}
\end{defn}

Then $\pi_2$ is a piece of the fundamental 2-groupoid: restrict the 0-layer to just one point, then take the resulting 2-layer.

\section{CW-complexes}

We can build interesting topological objects from simple ones.
For example, the torus can be constructed from one copy of $D^0$, two of $D^1$ and one of $D^2$ glued in a certain way.

A CW-complex (also known as a cell complex) is a topological object that can be build from various $D^n$ (we call $D^n$ an $n$-cell).
We do this iteratively as follows:

Let $X^0$ be some number of copies of $D^0$.
Then we define $X^i$ by
$$X^i = (X^{i-1}\cup \bigcup_{\alpha} D_{\alpha}^i)/\{\phi_\alpha^i\}$$
where $\phi_\alpha^i: \del D_\alpha^i\to X^{i-1}$.

So for the torus, $X_0$ is one copy of $D^0$, $X_1$ is a point with two loops from it, and $X_2$ is the torus.

CW-complexes have lots of nice properties, and close to every interesting topological object can either be written as one, or replaced by one with the same homotopy groups.
For example, we can calculate $\pi_1$ relatively simply from such a construction.
Also, it is easy to tell when two CW-complexes are homotopy equivalent:

\begin{defn}
	We say two topological spaces $X, Y$ are \emph{homotopy equivalent} if there exists some $f: X\to Y$ and $g: Y\to X$ such that $f\of g$ and $g\of f$ are both homotopic to $\operatorname{id}$.
\end{defn}

\begin{thm}[Whitehead]
	Let $X, Y$ be CW-complexes, with $f: X\to Y$ inducing an isomorphism on all homotopy groups of $X$ and $Y$.
	Then $f$ gives a homotopy equivalence between the two spaces.
\end{thm}

We can construct $\pi_1$ of the torus from its CW-complex construction.
We have $\pi_1(X_0)$ trivial, as $X_0$ is a point.
Then $\pi_1(X_1)$ is the free group on 2 generators as $X_1$ is two loops.
However, adding a copy of $D^2$ adds the relation $aba^{-1}b^{-1}$, so this is the abelian group on 2 generators, which is $\ZZ\oplus\ZZ$. 

\begin{thm}
	Given a CW-complex $X$, $\pi_1(X)$ is the same as the fundamental group of its 2-skeleton.
	In general, $\pi_n(X)$ is the same as $\pi_n$ of its $(n-1)$-skeleton.
\end{thm}

Next, we consider building the projective plane as a CW-complex.
This is not so easy to visualise or draw!

\begin{defn}
The \emph{projective plane} in two dimensions is $S^2$ quotiented by identifying pairs of opposite points.
\end{defn}

First, consider building the sphere as a cell complex.
(Again, this is easier to draw than describe.)

We have $X^0$ being a pair of points, then we add two copies of $D^1$ for $X^1$ to give a circle with two marked points, then we add two copies of $D^2$ for $X^2$: one for each hemisphere.

For the projective plane, we instead start with one point for $X^0$ (as the pair of points in the sphere are opposite each other so would be identified).
We then add one copy of $D^1$ to form a loop with one marked point for $X^1$, then add two copies of $D^2$ to this for $X^2$, which each connect to the loop by wrapping around twice.
This is the projective plane.

Building up the fundamental group, we have $\pi_1(X^0)$ trivial, then $\pi_1(X^1)$ is the free group on one generator $u$ (as we have only one loop to traverse) which is $\ZZ$.
Finally, adding the copies of $D^2$ adds the relation $u^2$, so we get $\ZZ_2$.

\begin{thm}
Every group is the fundamental group of some CW-complex.
\end{thm}

The proof is constructive. It looks like adding one loop for each generator, then adding 2-cells to induce relations.

This method also allows us to `truncate' the homotopy groups past $\pi_N$, making them trivial for all $\pi_n$ with $n > N$.
We just add high-dimensional cells to kill the generators of these higher homotopy groups.
This may require an infinite number of additional cells, but note we will be able to map the original cell complex into the constructed one.
(Apologies for how vague that got!)
\end{document}

